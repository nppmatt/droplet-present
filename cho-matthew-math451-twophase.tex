\documentclass[12pt]{beamer}

% \usepackage[margin=1in]{geometry}
\usepackage{parskip}
\usepackage{amssymb}
\usepackage{mathtools}
\usepackage{matlab-prettifier}
\usepackage{multimedia}

\author{Matt Cho}
\title{Simulating a Droplet Within a Two-phase Viscous Flow}
\date{}
\usetheme{Rochester}
\usecolortheme{lily}
\setbeamertemplate{navigation symbols}{}

\begin{document}
\maketitle

% NJIT
\begin{frame}
    \frametitle{Admin Things}
    \includegraphics[width=\textwidth]{img/8-slurm.png}
\end{frame}
\begin{frame}
    \frametitle{Admin Things}
    \includegraphics[width=0.8\textwidth]{img/9-njit.png}
\end{frame}

% Analytical Method vs Exploratory Analysis
% Modern Analysis vs "Post-Moore's Law"-Analysis
\begin{frame}
    \frametitle{CFD Methodology}
    \underline{Common Ground}

    We have a mental model. Is it closer to what we think is real or not?
\end{frame}
\begin{frame}
    \frametitle{CFD Methodology}
    \underline{Scientific Method}

    Analytical method where we go through an iterative process of:

    \textbf{Question $\rightarrow$ Hypothesis $\rightarrow$ Experiment $\rightarrow$ Refinement}

    (Engineering methodology is identical, but with cost constraints added.)
\end{frame}
\begin{frame}
    \frametitle{CFD Methodology}
    \underline{Scientific Method}

    \includegraphics[width=0.7\textwidth]{img/0-caltech-maxwell.png}
\end{frame}
\begin{frame}
    \frametitle{CFD Methodology}
    \underline{Computational Math ("Post-Moore's Law")}

    We have a virtual model written in the form of an algorithm. How do we make the most of it?
\end{frame}
\begin{frame}
    \frametitle{CFD Methodology}
    \underline{Computational Math ("Post-Moore's Law")}

    Process is fundamentally the same, but we now have 1000s of CPUs to do 
    what we want:

    \textbf{Question $\rightarrow$ Hypothesis $\rightarrow$ Experiment $\rightarrow$ Experiment $\rightarrow$ Experiment $\rightarrow$ Experiment $\rightarrow$ Experiment $\rightarrow$ \dots}
\end{frame}
\begin{frame}
    \frametitle{CFD Methodology}
    \underline{Computational Math ("Post-Moore's Law")}

    \includegraphics[width=0.7\textwidth]{img/1-fisher.jpg}
\end{frame}
\begin{frame}
    \frametitle{CFD Methodology}
    \underline{Computational Math ("Post-Moore's Law")}

    \includegraphics[width=0.7\textwidth]{img/2-sisyphus.jpg}
\end{frame}

% Introducing software framework, Basilisk
\begin{frame}
    \frametitle{Software Framework}
    \underline{Introducing Basilisk}

    General Purpose PDE Solver specializing in solving over adaptive meshes.

    Creator is Stéphane Popinet of Sorbonne Université (formerly Université Pierre-et-Marie-Curie) in Paris
\end{frame}
\begin{frame}
    \frametitle{Software Framework}
    \underline{Introducing Basilisk}

    \includegraphics[width=\textwidth]{img/3-hidman2023.jpg}
\end{frame}
\begin{frame}
    \frametitle{Software Framework}
    \underline{What is this really?}

    A solution to the Navier-Stokes equations governing flows,
    using the Volume-of-Fluid method to track where the fluid 
    interface is located in space.

    \includegraphics[width=\textwidth]{img/simplified-ns.png}
\end{frame}
\begin{frame}
    \frametitle{Software Framework}
    \underline{Code Library (Headers)}

    \includegraphics[width=0.8\textwidth]{img/5-headers.png}
\end{frame}
\begin{frame}
    \frametitle{Software Framework}
    \underline{Specifying Non-dimensional Numbers}

    \includegraphics[width=0.8\textwidth]{img/numbers.png}
\end{frame}
\begin{frame}
    \frametitle{Software Framework}
    \underline{Specifying Non-dimensional Numbers}

    \includegraphics[width=\textwidth]{img/numberslist.png}
\end{frame}
\begin{frame}
    \frametitle{Software Framework}
    \underline{Specifying Non-dimensional Numbers}

    \includegraphics[width=0.8\textwidth]{img/more-numbers.png}
\end{frame}
\begin{frame}
    \frametitle{Software Framework}
    \underline{Specifying Problem Geometry}

    \includegraphics[width=0.8\textwidth]{img/geometry.png}
\end{frame}
\begin{frame}
    \frametitle{Software Framework}

    \includegraphics[width=\textwidth]{img/4-drop.png}
\end{frame}
\begin{frame}
    \frametitle{Software Framework}
    \underline{What is this really?}

    A solution to the Navier-Stokes equations,
    using the Volume-of-Fluid method (VOF).

    Problem of analysing real droplet behavior is a combination of problems. In 
    \underline{isotropic conditions}, this is essentially solving for a 
    two-phase flow problem where the substrate is a solid boundary.

    We are interested in viscosity and advection (transport) more than 
    heat transfer, for now.
\end{frame}
\begin{frame}
    \frametitle{Software Framework}

    \includegraphics[width=1.75\textwidth]{img/6-twophase.png}
\end{frame}

% Showing off qualitative aspects of development.
\begin{frame}
    \frametitle{Current Work: 2-D Simulation}
    \underline{Some Problems and Solutions:}

    - Even for symmetric problems, mirroring about an axis can be problematic.
\end{frame}
\begin{frame}
    \frametitle{Current Work: 2-D Simulation}

    \includegraphics[width=0.7\textwidth]{img/11-pillowing.png}
\end{frame}
\begin{frame}
    \frametitle{Current Work: 2-D Simulation}
    \underline{Some Problems and Solutions:}

    - Even for symmetric problems, mirroring about an axis can be problematic.

    $\rightarrow$ Use a 2D model.
\end{frame}
\begin{frame}
    \frametitle{Current Work: 2-D Simulation}
    \underline{Some Problems and Solutions:}

    - Even for symmetric problems, mirroring about an axis can be problematic.

    $\rightarrow$ Use a 2D model.

    - At small length scales, numerical errors dominate the physics unless if 
    they are mitigated.
\end{frame}
\begin{frame}
    \frametitle{Current Work: 2-D Simulation}

    \includegraphics[width=0.8\textwidth]{img/cuttingwood.jpg}

    Making a table with 64 bits.
\end{frame}
\begin{frame}
    \frametitle{Current Work: 2-D Simulation}

    \includegraphics[width=0.7\textwidth]{img/10-assymetry.png}
\end{frame}
\begin{frame}
    \frametitle{Current Work: 2-D Simulation}
    \underline{Some Problems and Solutions:}

    - Even for symmetric problems, mirroring about an axis can be problematic.

    $\rightarrow$ Use a 2D model.

    - At small length scales, numerical errors dominate the physics unless if 
    they are mitigated.

    $\rightarrow$ Use an adaptive grid and tune it.
    
    $\rightarrow$ There is no avoiding going smaller without approximation, 
    which in itself is generally not a suitable solution.
\end{frame}
\begin{frame}
    \frametitle{Current Work: 2-D Simulation}
    \underline{Some Problems and Solutions:}

    - Even for symmetric problems, mirroring about an axis can be problematic.

    - At small length scales, numerical errors dominate the physics unless if 
    they are mitigated.
    
    - Experiment Design: Trying to match real conditions is very 
    computationally intensive, have to settle for simpler/coarser model with 
    respective changes in non-dimensional numbers.
\end{frame}

% Qualatative results???
\begin{frame}
    \frametitle{Results and Future Work}
    Currently have a fairly reasonable simulation of a "nanodroplet" (micro and
    nanometer length scale) that agrees with constants reported in literature:
    
    (Wang et. al. "Phase diagram for nanodroplet impact on solid surfaces" 2021).
\end{frame}
\begin{frame}
    \frametitle{Results and Future Work}
    \includegraphics[width=0.8\textwidth]{img/wang-3d.png}
\end{frame}
\begin{frame}
    \frametitle{Results and Future Work}
    Literature Example:

    \
    \fbox{
    \includegraphics[width=0.6\textwidth]{img/wang-constants.png}
    }

    We = Weber Number = $\dfrac{\text{drag forces}}{\text{cohesive forces}}$
\end{frame}
\begin{frame}
    \frametitle{Results and Future Work}
    Literature Example:

    \
    \fbox{
    \includegraphics[width=0.6\textwidth]{img/wang-constants.png}
    }

    Our Constants:

    Reynolds = 5.0

    Weber = $\dfrac{\rho v^2 l}{\sigma} 
    = \dfrac{(0.001)(2.0)^2(1.0)}{0.05}
    = 0.08$ (nice!)
\end{frame}
\begin{frame}
    \frametitle{Results and Future Work}
    \includegraphics[width=\textwidth]{img/wang-sim.png}

    Results appear to generally agree with this more detailed 3-D simulation.
\end{frame}
\begin{frame}
    \frametitle{Results and Future Work}
    \includegraphics[width=0.7\textwidth]{img/wang-high-weber.png}

    Simulation result w.r.t. Weber Number provides an explanation for phenomena 
    such as "finger" formation, droplet fragmentation, frozen pancake, etc.
\end{frame}
\begin{frame}
    \frametitle{Results and Future Work}
    \includegraphics[width=0.7\textwidth]{img/7-current.png}
\end{frame}
\begin{frame}
    \frametitle{Future Work (Maths)}
    - Currently have a fairly reasonable simulation of a "nanodroplet" (nanometer length scale) that agrees with constants reported in literature.

    - Natural thing to do would be to incrementally approach larger scale 
    (mm-scale drops WIP currently).

    - Mid-long term would ideally add heat transfer and Stefan model. \\
    $\rightarrow$ Not trivial, but another group doing liquid-gas transition.
\end{frame}
\begin{frame}
    \frametitle{Future Work (Computational)}
    The way forward is simply applying known computer 
    programming/engineering techniques to make computational problem tractable.

    $\rightarrow$ Use GPUs, automatic check-in/compilation, MPI, etc.

    More exotic solutions include Physics-informed Neural Networks (PINNs).

    $\rightarrow$ Potentially provides a framework that can probabilistically 
    approximate chaotic systems.
\end{frame}

\end{document}
    